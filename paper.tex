\documentclass{easychair}
\usepackage{microtype}

\title{Specifying BYOD Policies with Authorization Logic}
\titlerunning{Specifying BYOD Policies with Authorization Logic}
\author{
  Joseph Hallett\inst{1}
  \and
  David Aspinall\inst{1}
}
\authorrunning{Hallett and Aspinall}

\institute{
  University of Edinburgh
}

% AppPAL and Listing formatting
\lstdefinelanguage{AppPAL}{%
  morekeywords={if,inf,says,where,true,false},
  otherkeywords={can-act-as,can-say},
  sensitive=true,
  morestring=[b]',
  literate={\ inf\ }{{$\infty$}}5
}[keywords,strings]
\lstset{%
  basicstyle=\ttfamily\footnotesize{},
  stringstyle=\sffamily\footnotesize{},
  keywordstyle=\ttfamily\slshape\footnotesize{},
  language=AppPAL,
  columns=flexible,
  frame=single
}

\newcommand{\dotdotdot}[1]{$\[\cdots\]$}

\begin{document}
\maketitle

\begin{abstract}
  % State problem
  Corporate BYOD policies are imprecisely specified using natural language.
  % Why is it interesting
  This creates ambiguity and makes detecting issues, and comparing policies harder.
  % What does my solution achieve
  We show how AppPAL (an authorization logic for mobile app installation policies) can describe BYOD policies formally.
  % What follows from solution
  This suggests that BYOD policies can be enforced automatically, and with greater precision, than by relying on users.
\end{abstract}

\section{Introduction}
% Corporate policies are poorly specified
% We have used an authorization to specify them and this is interesting

Employees are increasingly bringing their own devices into their workplaces.
This can present a challenge for corporate IT departments:
  they need to secure their networks and data but as they do not provide the employees with the devices they have limited control over what software runs and how the devices are configured.
As a means of controlling the devices, IT departments have taken to publishing mobile device policies.
Employees must agree to follow these policies if they want to use their personal devices in work.
Often these policies are specified using natural language.
This is problematic as it can create ambiguity in how the policy should be enforced, and in how the IT department can confirm that the policy was followed.
There may be issues with \emph{completeness} and \emph{consistency} in these policies, which can lead to problems in enforcement.

Logics of authorization are languages designed to express rules about permissible actions.
\emph{AppPAL}~\cite{Hallett:2016vu} is a language designed to express policies about which mobile apps to use and install.
AppPAL is an instantiation of Becker~et~al{.}'s SecPAL~\cite{Becker:2006vh} and can express complex delegation relationships as well as make use of external programs and constraints such as static analysis tools, whilst being decidable and having formal semantics.
We believe AppPAL can also be used to help specify BYOD policies. 
We present in Section~\ref{sec:translation} excerpts from the translation into AppPAL of three BYOD policies: one from a security policy template from the SANS institute to aid companies in developing their own policies~\cite{Guerin:2008we}, another from an NHS hospital trust describing what their doctors may do with their phones and their role in the treatment of patients~\cite{Kennington:2014tr}, finally a simplistic policy used in practice from the University of Edinburgh~\cite{Williamson:2015tr}.

% TODO: We found that...

\section{Translating Corporate BYOD Policies}
\label{sec:translation}

It is common to find rules in policies describing what apps can be used on devices.
This is modeled in AppPAL through the \texttt{isInstallable} predicate.
The Edinburgh policy only recommends that apps come from a reputable source though what it means to be \emph{reputable} is left undefined. 
\begin{lstlisting}[title={\footnotesize\textbf{Edinburgh}:~\itshape ``Only download applications (‘apps’) or other software from reputable sources.''}]
'edinburgh-policy' says Store can-say App isInstallable if Store sells(App), Store isReputable.
\end{lstlisting}
Whereas the SANS policy require the IT department maintain white and black lists of acceptable apps:
\begin{lstlisting}[title={\footnotesize\textbf{SANS}:~\itshape ``The IT Department maintains a list of allowed and unauthorzed applications and makes them available to users on the intranet.''}]
'sans-policy' says 'it-department' can-say App isInstallable.
\end{lstlisting}
This rule is replicated later in the policy by another recommendation that says the same:
\begin{lstlisting}[title={\footnotesize\textbf{SANS}:~\itshape ``Only approved third party applications can be
installed on handhelds. The approved list can be
obtained by contacting the IT department, or should be
available on the intranet.''}]
'sans-policy' says 'it-department' can-say App isInstallable.
\end{lstlisting}
The NHS policy makes a distinction between \emph{clinical} and \emph{non-clinical} apps with each requiring approval through a different group before final approval is given by the \emph{Integrated Governance Committee}.  These rules are only theoretical however as no group has ever approved an app.
\begin{lstlisting}[title={\footnotesize\textbf{NHS}:~\itshape ``%
Apps for work usage must not be downloaded onto corporately issued mobile devices (even if approved on the NHS apps store) unless they have been approved through the following Trust channels: Clinical apps; at the time of writing there are no apps clinically approved by the Trust for use with patients / clients. However, if a member of staff believes that there are clinical apps or other technologies that could benefit their patients / clients, this should be discussed with the clinical lead in the first instance and ratification should be sought via the Care and Clinical Policies Group. A clinical app should not be used if it has not been approved via this group. Business apps; at the time of writing there are no business (i.e., non-clinical) apps approved by the Trust for use other than those preloaded onto the device at the point of issue. However, if a member of staff believes that there are apps or other technologies that could benefit their non-clinical work, ratification of the app must be sought via the Management of Information Group (MIG). An app should not be used if it has not been approved via this group. Following approval through Care and Clinical Policies and / or MIG, final approval will be required through Integrated Governance Committee.''}]
'nhs-policy' says App isInstallable if App isUsable, App isFinallyApproved.
'nhs-policy' says 'cacpg' can-say App isUsable if App isUsableClinically.
'nhs-policy' says 'mig' can-say App isUsable if App isUsableNonClinically.
'nhs-policy' says 'igc' can-say App isFinallyApproved.
\end{lstlisting}

In each case the rules for deciding whether an employee could install an app took the form of a delegation to one or more third-parties to approve apps.
By translating the policies into AppPAL problems become immediately apparent: the Edinburgh policy is incomplete as there was no way to determine what stores were reputable, the SANS policy contains duplicated rules, and the NHS policy is redundant as no app had ever been approved for use. 
In each case the AppPAL rule was no longer than the natural language policy, and in the case of the NHS policy considerably shorter and clearer.

Some rules also appear to be common between different policies.
For example both the NHS and SANS policies contain rules on the use of mobile devices in cars.
The NHS policy requires employees do not use devices in cars, and recommends against using them even with hands-free equipment:
\begin{lstlisting}[title={\footnotesize\textbf{NHS}:~\itshape ``%
For safety reasons, Trust staff must not use a hand held mobile device whilst driving any vehicle. It is illegal to do so. Please refer to the most up-to-date information via the Highways Agency.  For safety reasons, Trust staff must not use a hand held mobile device whilst driving any vehicle. It is illegal to do so. Please refer to the most up-to-date information via the Highways Agency.''}]
'nhs-policy' says Device mustNotOperate 
  if Device isOwnedBy(Owner)
  where inCar(Device) = true, isDriving(Owner) = true.
\end{lstlisting}
Whereas the SANS policy requires the use of hands-free devices when in a car.
\begin{lstlisting}[title={\footnotesize\textbf{SANS}:~\itshape ``Conducting telephone calls or utilizing handhelds while driving can be a safety hazard. Drivers should use handhelds in hand only while parked or out of the vehicle.
If employees must use a handheld device while driving, ⟨Company⟩ requires the use of hands-free headset devices.''}]
'sans-policy' says Device mustNotOperate 
  if Device isOwnedBy(Owner)
  where inCar(Device) = true, isDriving(Owner) = true, isUsingHandsFree(Device) = false.
\end{lstlisting}
The similarity between the two rules is immediately apparent when specified in AppPAL: the SANS version has an additional constraint but they are otherwise identical.
Other similarities appear when looking at how devices connect to networks.  The Edinburgh policy recommends disabling automatic access to open unsecured Wi-Fi networks, where as SANS says its okay if the access point is secured properly or a VPN is used:
\begin{lstlisting}[title={\footnotesize\textbf{Edinburgh}:~\itshape ``Control your devices connections by disabling automatic connection to open, unsecured Wi-Fi networks.''}]
'edinburgh-policy' says Device canConnectTo(AP) if AP isSecure.
\end{lstlisting}
\begin{lstlisting}[title={\footnotesize\textbf{SANS}:~\itshape ``If mobile workers do require connectivity through public, open, or untrusted WLAN, then users MUST use WLANs using, if available and in this order: WPA(2) encryption, WEP 256 bits (or 128 bits), or finally open networks if nothing else is available. Users connected to data networks in an open environment MUST use a VPN connection.''}]
'sans-policy' says Device canConnectTo(AP) if AP isSecure.
'sans-policy' says AP isSecure if AP canUseWPA2.
'sans-policy' says AP isSecure if AP canUseWEP256.
'sans-policy' says AP isSecure if AP canUseWEP128.
'sans-policy' says Device canConnectTo(AP) if Device hasVPN(VPN) where connect(VPN) = true.
\end{lstlisting}
Similarities between policies suggest that one could define a common schema of BYOD predicates, which would allow different policies to be compared \emph{precisely} using standard terms.

\section{Conclusions}
% Related work:
%   Existing systems for enforcement (mobile iron)
%   Other work translating policies into formal language
% Future work:
%   Full case study translating and implementing a policy into AppPAL
% Conclusion:
%   Natural language policies don't work
%   Formal specification helpful.


We have shown...


\bibliographystyle{plain}
\begin{thebibliography}{1}
\bibitem{Becker:2006vh}
M.Y. Becker, C~Fournet, and A~D Gordon.
\newblock {SecPAL: Design and semantics of a decentralized authorization language}.
\newblock {\em Computer Security Foundations}, 2006.

\bibitem{Guerin:2008we}
N.R.C. Gu{\'e}rin.
\newblock {Security Policy for the use of handheld devices in corporate environments},
\newblock {\em SANS}.%, 2008.

\bibitem{Hallett:2016vu}
J. Hallett and D. Aspinall.
\newblock {AppPAL for Android}.
\newblock {\em Engineeering Secure Software and Systems}, 2016.

\bibitem{Kennington:2014tr}
G. Kennington, et~al.
\newblock {Mobile Devices Policy},
\newblock {\em Torbay and Southern Devon Health Care NHS Trust}.%, 2014.

\bibitem{Williamson:2015tr}
D. Williamson, A. Grzybowski, and S. Graham.
\newblock {BYOD Policy: Use of Personally Owned Devices for University Work},
\newblock {\em University of Edinburgh}.%, 2015.
\end{thebibliography}
\end{document}
